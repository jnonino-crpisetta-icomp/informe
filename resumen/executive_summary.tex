%%%%%%%%%%%%%%%%%%%%%%%%%%%%%%%%%%%%%%%%%%%%%%%%%%%%%%%%%%%%%%%%%%%%%%%%%%%%%%%%%%%%%
%																					%
%	TRABAJO: Proyecto Integrador													%
%																					%
%		Titulo: 	Desarrollo de IP cores con procesamiento de Redes de Petri 		%
%					Temporales para sistemas multicore en FPGA						%
%																					%
%		Autores:	Juli�n Nonino													%
%					Carlos Renzo Pisetta											%
%		Director:	Orlando Micolini												%
%																					%
%	Capitulo: Executive Summary														%	
%	Archivo: executive_summary.tex													%
%																					%
%%%%%%%%%%%%%%%%%%%%%%%%%%%%%%%%%%%%%%%%%%%%%%%%%%%%%%%%%%%%%%%%%%%%%%%%%%%%%%%%%%%%%

\chapter*{Executive Summary}
	\label{chap:executive_summary}
	
	Since the modern systems have multiple process and threads running at the same time, 
	they need a mechanism to synchronize them. Today, these mechanisms are implemented in 
	software and do not have support in the hardware. Semaphores and monitors are examples 
	of these control structures.
	\\
	
	On the oher hand, the most used method to model concurrent systems are state machines 
	with model checking, propositional logic, etc. The problem with this is that the distance 
	between the model and the implementation is too big so, it is difficult to verify in the 
	implementation the properties verified in the model.
	\\
	
	For these mentioned reasons, researchers in the \emph{Computer Architecture Laboratory} 
	are working in \textbf{\emph{Petri Nets}}. This is a graphical way to model concurrent 
	systems and has a formal mathematical background. The new discovery made in the laboratory 
	is that the Petri Nets can be \textbf{\emph{executed}}, not only used for simulations. This 
	way, \emph{the distance between the model and the implementation does not exist}. In the 
	implementation are fulfilled all the properties verified in the model because the implementation 
	\textbf{\emph{IS}} the model, \textbf{\emph{IS}} the Petri Net.
	
	Then, the next step is to implement in hardware a module capable of execute Petri Nets, this is 
	the fourth project in this line of work. The first of them demonstrated that the implementation 
	in hardware of a Petri Nets processor was possible \cite{paillertejeda}. Then, the second project 
	was the simulation of multicore systems synchronized by Petri Nets  \cite{baldoniromano} and the third, 
	was the implementation of the first Petri Nets processor; it can execute simple nets and nets with 
	inhibitor arcs \cite{galliapereyra}.
	\\	
	
	Finally, in this project, we translate the previous version from \emph{VHDL} to \emph{Verilog} and 
	improve the design to allow the addition of new functionalities such as bounds in the places of the 
	net, automated firing of transitions and, the most important feature added, \textbf{\emph{the capability 
	to manage time}}. For this reason, we develop two \emph{IP cores}, one to execute \textbf{\emph{Time Petri 
	Net}}, and the second to process \textbf{\emph{Timed Petri Nets}}, these are the two existing semantics to
	manage time in Petri Nets. 
	
	At the testing stage, we used classic problems of concurrency like, multiple writers trying to write in the 
	same variable, the \emph{dining philosopher's} problem and one inventing by us, the \emph{tables factory}. With 
	these tests, we verified the functionality of the \emph{IP cores}. Besides, we implement the resolution of the 
	same problems using semaphores as synchronization method. Then, comparing execution time measurements, we found 
	that the use of the Petri Nets processor is from $15\$$ to $30\%$ faster than the implementations that use 
	semaphores for synchronization, reaching more than $70\%$ of improvement according of the problem.
	
	Also, it has been observed that to implement a system with Petri nets is much simpler than using other synchronization 
	mechanisms and also has greater versatility as it allows three different methods to wait for the required synchronization 
	conditions (transitions triggered).

	Note that the IP cores developed were tested with three problems with very different characteristics and all the case 
	were successful.

	