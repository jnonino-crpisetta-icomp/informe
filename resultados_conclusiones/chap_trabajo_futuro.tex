%%%%%%%%%%%%%%%%%%%%%%%%%%%%%%%%%%%%%%%%%%%%%%%%%%%%%%%%%%%%%%%%%%%%%%%%%%%%%%%%
%	TRABAJO: Proyecto Integrador
%		Titulo: 	Desarrollo de IP cores con procesamiento de Redes de Petri 	
%					Temporales para sistemas multicore en FPGA					
%		Autores:	Juli�n Nonino												%					Carlos Renzo Pisetta										%		Director:	Orlando Micolini											
%%%%%%%%%%%%%%%%%%%%%%%%%%%%%%%%%%%%%%%%%%%%%%%%%%%%%%%%%%%%%%%%%%%%%%%%%%%%%%%%

\chapter{Trabajo futuro}
	\label{chap:chap_trabajo_futuro}

	En base a los resultados obtenidos con los IP cores desarrollados, inmediatamente surgen temas en los que se debe trabajar para generar nuevas y mejores versiones de los mismos. 
	\\
	
	La principal mejora en la cual se debe continuar trabajando es refinar la implementaci�n en Verilog y generar nuevas ideas para reducir el �rea que ocupan los IP cores al sintetizarlos para FPGA. Con respecto a esto, se puede trabajar optimizando los datos que deben cargarse para reducir el tama�o de las estructuras necesarias para el algoritmo de ejecuci�n de Redes de Petri. Tambi�n, se pueden realizar investigaciones con respecto a la posibilidad de utilizar memorias externas para almacenar todos estos datos, dejando en la FPGA s�lo la l�gica de operaci�n. Con respecto a este �ltimo punto, para optimizar el aprovechamiento de memoria, se deben realizar estudios que lleven a un mejor aprovechamiento de la localidad temporal y espacial de los datos. De esta manera, se generan estructuras de datos m�s peque�as dentro del procesador de Redes de Petri para poder obtener resultados r�pidamente y, se almacenan en memoria aquellos datos con menor probabilidad de utilizaci�n.
	\\

	Trabajando en las mejoras antes mencionadas, es posible construir sistemas de gran envergadura gracias a un procesador capaz de ejecutar Redes de Petri de gran tama�o, de manera tal que casi cualquier problema puede ser modelado e implementado a trav�s de la ejecuci�n de Redes de Petri.
	\\
	
	Por otro lado, una mejora que se esta llevando a cabo en el Laboratorio de Arquitectura de Computadoras, es el desarrollo un driver que permita utilizar el procesador de Redes de Petri desde lenguajes de alto nivel y sistemas operativos como Linux, Windows, etc.
	\\
	
	En este trabajo, se ha implementado un procesador capaz de resolver Redes de Petri de bajo nivel como son: Redes de Petri con Arcos Inhibidores, Redes de Petri con Tiempo y Redes de Petri Temporizadas. Existen redes de alto nivel que aumentan considerablemente el poder de modelado de las Redes de Petri. Entre ellas, est�n las Redes de Petri Bien Formadas y las Redes de Petri Coloreadas. En estas redes los tokens ya no son todos del mismo tipo, existen diversos tipos de tokens lo que posibilita trabajar con diferentes tipos de datos. De esta manera, se agregan muchos sistemas al conjunto de aquellos que pueden ser modelados con Redes de Petri. Con respecto a este tema, en el Laboratorio de Arquitectura de Computadoras ya se est�n investigando posibilidades para dise�ar e implementar un IP core capaz de ejecutarlas.
