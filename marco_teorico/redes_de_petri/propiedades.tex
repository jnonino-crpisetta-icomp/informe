%%%%%%%%%%%%%%%%%%%%%%%%%%%%%%%%%%%%%%%%%%%%%%%%%%%%%%%%%%%%%%%%%%%%%%%%%%%%%%%%
%	TRABAJO: Proyecto Integrador
%		Titulo: 	Desarrollo de IP cores con procesamiento de Redes de Petri 	
%					Temporales para sistemas multicore en FPGA					
%		Autores:	Juli�n Nonino												%					Carlos Renzo Pisetta										%		Director:	Orlando Micolini											
%%%%%%%%%%%%%%%%%%%%%%%%%%%%%%%%%%%%%%%%%%%%%%%%%%%%%%%%%%%%%%%%%%%%%%%%%%%%%%%%

% Path im�genes: ./marco_teorico/redes_de_petri/img
% Nombre predeterminado im�genes: petrixx
%	xx es el numero de imagen

\section{Propiedades de las Redes de Petri}
	\label{sec:propiedades_petri}

	Las Redes de Petri tienen propiedades especiales que se detallar�n a continuaci�n.
	
	\subsection{Redes de Petri limitadas}
		
		Sea la Red de Petri $PN = {P, T, I^-, I^+, m_0}$, se dice que una plaza $p$ es \textbf{\emph{k-limitada}} si existe un valor $k$ tal que para todo marcado $m$ perteneciente al conjunto de marcados de la Red de Petri $PN$ se cumple que el valor de marcado de la plaza $p$ es menor o igual que $k$.
		\begin{equation}
			\exists k / \forall m \in marcados(PN) : m(p) \leq k
			\label{eq:red_limitada}
		\end{equation}
		
		El arreglo $marcados(PN)$ representa todos los marcados posibles de la Red de Petri. Se dice que una Red de Petri es \emph{k-limitada} si todas sus plazas son \emph{k-limitadas}.
		
	\subsection{Redes de Petri seguras}
	
		Se dice que una red de Petri PN es \textbf{\emph{segura}} si y s�lo si todas sus plazas son \textbf{\emph{1-limitadas}}. Esto implica que una transici�n no puede ser disparada si alguna de sus plazas de llegada est� ocupada.
		
	\subsection{Redes de Petri c�clicas}
			
		Se dice que una Red de Petri es \textbf{\emph{c�clica}} si existe siempre una sucesi�n de marcados que lleve desde cualquier marcado $m$ al marcado inicial $m_0$.
	
	\subsection{Redes de Petri repetitivas}

		Se dice que una Red de Petri es \textbf{\emph{repetitiva}} si existe siempre una secuencia de disparos que lleve desde un marcado $m$ hasta el mismo marcado $m$.

	\subsection{Redes de Petri conservativas}

		Una Red de Petri es \textbf{\emph{conservativa}} si se cumple que para todo marcado $m$ perteneciente al conjunto de marcados de la red, la cantidad de marcas de $m$ es igual a la cantidad de marcas de $m_0$. Es decir, se conserva la cantidad de tokens.

	\subsection{Propiedad de vivacidad (liveness)}

		Se dice que una Red de Petri esta \textbf{\emph{viva}} si y s�lo si en todo momento sus transiciones pueden ser disparadas o si existe una secuencia de disparos v�lidos que lleven a que la transici�n pueda dispararse. 

	\subsection{Propiedades de bloqueo (deadlock)}
	
		Un \textbf{\emph{bloqueo}} en una Red de Petri implica que una transici�n nunca podr� ser disparada independientemente de la secuencia de disparos que se aplique. Recordando la propiedad anterior, se puede decir que una Red de Petri viva garantiza la ausencia de bloqueos. Este concepto est� sustentado en la ausencia de \emph{marcados sumideros}. Un marcado sumidero es aquel marcado a partir del cual ninguna transici�n puede ser disparada.
		
	\subsection{Alcanzabilidad}

		La alcanzabilidad determina si un marcado $m$ es alcanzable desde un marcado inicial $m_0$, es decir, determina si $m$ pertenece al conjunto de marcados de la Red de Petri. Se dice que un marcado $m$ es alcanzable desde $m_0$ si y s�lo si existe una secuencia de disparos que lleven desde $m_0$ hasta $m$. Para determinar si un marcado $m$ es alcanzable, se utiliza la ecuaci�n de estado de una Red de Petri de la siguiente manera:
		\begin{equation}
			m = m_0 + I � \sigma
			\label{eq:alcanzabilidad}
		\end{equation}
		D�nde $\sigma$ es una secuencia de disparos.
				
		Si no fuera posible encontrar una secuencia de disparos, se dice que el marcado $m$ es \emph{inalcanzable}. Si existieran m�ltiples soluciones, es decir, varias secuencias de disparos posibles para alcanzar el marcado $m$, se dice que la Red de Petri se comporta de manera \emph{NO determin�stica}. Hay que notar que la soluci�n de la ecuaci�n no indica el orden en que las transiciones deben ser disparadas, por lo que, �ste debe ser encontrado ejecutando todas las secuencias posibles. Por ello, es posible que a pesar de encontrar una secuencia de disparos, no exista un ordenamiento de los mismos que permita alcanzar la marca $m$ y sea aplicable a la Red de Petri en cuesti�n.
		
		\subsubsection{Grafo de alcanzabilidad}
				
			El grafo de alcanzabilidad de una Red de Petri representa el conjunto de todos los marcados alcanzables desde el marcado inicial $m_0$. Consiste en un grafo en forma de �rbol en el que cada nodo es un marcado alcanzable de la red y se conectan mediante arcos etiquetados con la transici�n que se dispara para pasar de un marcado a otro.
			
			Partiendo del estado inicial $m_0$ se generan todos los estados alcanzables desde �ste mediante un disparo de cada una de las transiciones sensibilizada. A partir de cada estado, se vuelve a repetir el proceso, apareciendo, en consecuencia, un grafo en forma de �rbol que representa todos los estados del sistema.
			
			Por ejemplo, sea la siguiente Red de Petri, se plantear� el grafo de alcanzabilidad de la misma.
				
			\begin{figure}[H]
				\centering
				\includegraphics[width=0.5\linewidth]{./marco_teorico/redes_de_petri/img/Petri02}
				\caption{Red de Petri de Ejemplo para Grafo de alcanzabilidad}
				\label{fig:Petri02}
			\end{figure}
				
			El grafo de alcanzabilidad de la red de la Figura \ref{fig:Petri02}, comenzar�a de la siguiente manera:
			\begin{figure}[H]
				\centering
				\includegraphics[width=0.65\linewidth]{./marco_teorico/redes_de_petri/img/Petri03}
				\caption{Ejemplo de grafo de alcanzabilidad}
				\label{fig:Petri03}
			\end{figure}
			El proceso se detiene en los marcados repetidos, que en la Figura \ref{fig:Petri03} aparecen recuadrados.
