%%%%%%%%%%%%%%%%%%%%%%%%%%%%%%%%%%%%%%%%%%%%%%%%%%%%%%%%%%%%%%%%%%%%%%%%%%%%%%%%%%%%%
%																					%
%	TRABAJO: Proyecto Integrador													%
%																					%
%		Titulo: 	Desarrollo de IP cores con procesamiento de Redes de Petri 		%
%					Temporales para sistemas multicore en FPGA						%
%																					%
%		Autores:	Juli�n Nonino													%
%					Carlos Renzo Pisetta											%
%		Director:	Orlando Micolini												%
%																					%
%	Parte: Marco Teorico															%
%	Capitulo: Redes de Petri														%
%	Seccion: Extensiones de las Redes de Petri										%	
%	Archivo: extensiones.tex														%
%																					%
%%%%%%%%%%%%%%%%%%%%%%%%%%%%%%%%%%%%%%%%%%%%%%%%%%%%%%%%%%%%%%%%%%%%%%%%%%%%%%%%%%%%%

% Path Imagenes: ./marco_teorico/redes_de_petri/img
% Nombre predeterminado imagenes: petrixx
%	xx es el numero de imagen

\section{Extensiones de las Redes de Petri}
	\label{sec:extensiones}

	Las Redes de Petri analizadas hasta el momento, pueden ser extendidas agregando arcos inhibidores para 
	considerar la ausencia de tokens como condici�n de sensibilizaci�n de una transici�n 
	(\textbf{\emph{Redes de Petri con Arcos Inhibidores}}), marcas de tiempo para determinar intervalos en los cuales 
	una transici�n puede ser disparada (\textbf{\emph{Redes de Petri con Tiempo}}), o valores de duraci�n en las 
	transiciones(\textbf{\emph{Redes de Petri Temporizadas}}), probabilidades de disparo de transiciones 
	(\textbf{\emph{Redes de Petri Estoc�sticas}}) o cualquier combinaci�n entre ellas.
	
	En la Figura \ref{fig:Petri14}, extra�da del libro \cite{diaz_petri}, se observa la evoluci�n 
	de estas extensiones. Se encuentran marcadas, aquellas que se analizar�n en este trabajo. Las \emph{Redes de 
	Petri con Arcos Inhibidores} est�n incluidas dentro de �tem \emph{Place-transition Petri nets 1962, 1969}.
		
	\begin{figure}[H]
		\centering
		\includegraphics[width=1\linewidth]{./marco_teorico/redes_de_petri/img/Petri14}
		\caption{Extensiones de las Redes de Petri}
		\label{fig:Petri14}
	\end{figure}	
		